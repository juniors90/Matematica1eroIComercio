\documentclass[10pt]{examdesign}
\OneKey
\usepackage[spanish]{babel}
\usepackage[utf8]{inputenc}
\usepackage[T1]{fontenc}
\usepackage{pifont}
\usepackage{graphicx}
\usepackage{float}
\usepackage{enumitem}
\usepackage{multicol}
\usepackage{multirow}
\usepackage[colorlinks=true,
breaklinks=true,
linkcolor=blue,
urlcolor=red,
bookmarksopen=true]{hyperref}
\usepackage[pdftex,dvipsnames]{xcolor}
\definecolor{upforestgreen}{rgb}{0.0, 0.27, 0.13}
\definecolor{dukeblue}{rgb}{0.0, 0.0, 0.61}

%----------------------------------------------------------------------------------------------
% Si desea utilizar \@@line para definir su propio encabezado de examen o palabras del encabezado, 
% asegúrese de usar \makeatletter y \makeatother en los lugares apropiados, de lo contrario 
% podría obtener errores.
%-----------------------------------------------------------------------------------------------

\makeatletter
%----------------------------------------------------------------------------------------------%
\begin{examtop}
	\@@line{\parbox{3in}{\classdata \\[0.5cm]
			\textcolor{upforestgreen}{\textbf{\underline{T.P.N$^\circ$}}~\fbox{\textsc{6}}} \examtype}
		%                  ^^^^^^
		\hfill
		\parbox{3in}{\normalsize \namedata}}
	\bigskip
\end{examtop}
%----------------------------------------------------------------------------------------------%
\def\namedata{\textcolor{upforestgreen}{\textbf{Estudiante}}:\hrulefill        \\[\namedata@vspace]
	\textcolor{upforestgreen}{\textbf{Curso y División}}: 1er año, I   \\[\namedata@vspace]
	\textcolor{upforestgreen}{\textbf{Profesor}}: Ferreira, Juan David \\[\namedata@vspace]
	\textcolor{upforestgreen}{\textbf{Fecha de Entrega}}:\hrulefill 
}
%----------------------------------------------------------------------------------------------% 
\begin{keytop}%
	\@@line{\hfill \Huge\texttt{\textcolor{upforestgreen}{Respuestas 
				Trabajo Práctico N$^\circ$~\fbox{\textsc{6}}}} \hfill}
	\bigskip
\end{keytop}%
%----------------------------------------------------------------------------------------------%
\makeatother
\examname{\textcolor{upforestgreen}{\underline{\textbf{Resolución de Problemas.}}}}

\SectionPrefix{\textcolor{upforestgreen}{\textbf{Sección \arabic{sectionindex}}.} \space}
\Fullpages
\ContinuousNumbering
\DefineAnswerWrapper{}{}
\NumberOfVersions{1}
\class{{\textcolor{upforestgreen}{\large\textbf{E.P.E.S. Nro 51 ``J. G. A.''}}\\[0.5cm]
		\textcolor{upforestgreen}{{\large \textbf{Matemática}}}}}
%----------------------------------------------------------------------------------------------%
\begin{document}
	%-------------------------------       SHORT ANSWER        ------------------------%
	\begin{shortanswer}[title={\textit{Receta que le dijo la Abuela Pocha a Ana y
				Juan.}},
		rearrange=no,resetcounter=no]
		\vspace*{-0.2cm}
		\begin{figure}[!h]
			\begin{minipage}[b]{0.3\textwidth}% --- 30% de la página
				\begin{center}% Figuras: ver capítulo 5
					\includegraphics{nena_chipa.png}
				\end{center}
			\end{minipage}
			\hfill
			\begin{minipage}[b]{0.65\textwidth} % --- 65% de la página
				\textbf{\textit{\textcolor{upforestgreen}{Ingredientes}}}:
				\begin{itemize}
					\item 300 g de cebolla picada,
					
					\item  1 kg (1000 g) de choclo
					cocido desgranado,
					
					\item 200 g de grasa de cerdo,
					
					\item 300 g de queso,
					
					\item  5 huevos batidos,
					
					\item 1 taza de leche,
					
					\item sal y pimienta a gusto
				\end{itemize}
			\end{minipage}
		\end{figure}
		\vspace{0.2cm}
		\textbf{\textit{\textcolor{upforestgreen}{Preparación}}}:
		\vspace{0.2cm}
		Saltear la cebolla picada en la grasa, agregar la leche y dejar cocinar $10$ minutos más, retirar del fuego y agregar el queso, los huevos batidos y el choclo. Mezclar bien y colocar en una asadera enmantecada, llevar al horno caliente y sacar cuando está dorado por encima. (\textcolor{dukeblue}{Texto de la receta de \textit{Chipa Guazú} extraído del libro formoseño “Gastronomía Formoseña". SANTANDER, Jorge M}).
		%-----------------------------------------------------------------------------------------------------------------------------------------------------------------------------------------------------------------------------------------------------------------------------------------------------------------%
		\begin{question}
			Completá el siguiente cuadro con las calorías que aporta cada ingrediente, en la receta de la abuela Pocha
			\begin{center}
				\begin{tabular}{|c|c|c|}
					\hline 
					\multirow{2}{*}{Ingredientes} &\multirow{2}{*}{Calorías }&
					Calorías en
					\\
					&&la receta de la abuela\\\hline  
					Cebolla frita (por cada $100g$)                  & $258$ &     
					\\\hline
					Choclo cocido (por cada $100g$)                  & $115$ &     
					\\\hline
					Grasa de cerdo (manteca de cerdo por cada $100g$)& $115$ &     
					\\\hline
					Queso (por $100g$)                               & $902$ &     
					\\\hline
					Huevo cocido (por $1$ unidad)                    & $310$ &     
					\\\hline
					Leche ($1$ taza)                                 & $155$ &     
					\\\hline
					Total                                            &       &     
					\\\hline
				\end{tabular}
			\end{center}
			\begin{answer}
				\textbf{Repuesta}: completar acá su respuesta. No se olviden de eso.
			\end{answer}
		\end{question}
		%--------------------------------------------------------------------------------------%
		\begin{question}
			¿Qué ingrediente tiene más calorías?.
			\begin{answer}
				\textbf{Repuesta}: completar acá su respuesta. No se olviden de eso.
			\end{answer}
		\end{question}
		%--------------------------------------------------------------------------------------%
		\begin{question}
			¿Cómo hiciste para completar la tercera columna?.
			\begin{answer}
				\textbf{Repuesta}: completar acá su respuesta. No se olviden de eso.
			\end{answer}
		\end{question}
		%--------------------------------------------------------------------------------------%
		\begin{question}
			¿Cuántas calorías hay en la segunda columna?.
			\begin{answer}
				\textbf{Repuesta}: completar acá su respuesta. No se olviden de eso.
			\end{answer}
		\end{question}
		%--------------------------------------------------------------------------------------%
		\begin{question}
			¿Cuántas calorías hay en la tercera columna?.
			\begin{answer}
				\textbf{Repuesta}: completar acá su respuesta. No se olviden de eso.
			\end{answer}
		\end{question}
		%--------------------------------------------------------------------------------------%
		\begin{question}
			¿De cuánto es la diferencia de calorías entre ambas columnas?.
			\begin{answer}
				\textbf{Repuesta}: completar acá su respuesta. No se olviden de eso.
			\end{answer}
		\end{question}
		%--------------------------------------------------------------------------------------%
		\begin{question}
			Si la receta es para obtener una asadera de chipa guazú, y la Abuela Pocha suele dividir
			la asadera en 6 porciones iguales ¿cuántas calorías habrá en una porción de chipa?.
			\begin{answer}
				\textbf{Repuesta}: completar acá su respuesta. No se olviden de eso.
			\end{answer}
		\end{question}
		%--------------------------------------------------------------------------------------%
	\end{shortanswer}
	%------------------------------------------------------------------------------------------%
\end{document}