\documentclass[12pt]{examdesign}
\usepackage[spanish]{babel}
\OneKey
\usepackage[utf8]{inputenc}
\usepackage[T1]{fontenc}
\usepackage{amsmath}
\usepackage{pifont}
%-----------------------------------------------------------------------------------------------
%\usepackage{gfsartemisia-euler}
\usepackage{graphicx}
\usepackage{float}
\usepackage{amscd}
\usepackage{amsfonts}
\usepackage{amssymb}
\usepackage{mathtools}
\usepackage{amsthm}
\usepackage[all]{xy}
\usepackage{enumitem}
\usepackage{multicol}
\usepackage{verbatim}
\usepackage[colorlinks=true,
breaklinks=true,
linkcolor=blue,
urlcolor=red,
bookmarksopen=true]{hyperref}
\usepackage[pdftex,dvipsnames]{xcolor}
\definecolor{aqua}{rgb}{0.0, 1.0, 1.0}
\definecolor{caribbeangreen}{rgb}{0.0, 0.8, 0.6}
\definecolor{tealgreen}{rgb}{0.0, 0.51, 0.5}
\definecolor{upforestgreen}{rgb}{0.0, 0.27, 0.13}
\definecolor{napiergreen}{rgb}{0.16, 0.5, 0.0}
\definecolor{capri}{rgb}{0.0, 0.75, 1.0}
\definecolor{calpolypomonagreen}{rgb}{0.12, 0.3, 0.17}
\definecolor{azure(colorwheel)}{rgb}{0.0, 0.5, 1.0}
\definecolor{dukeblue}{rgb}{0.0, 0.0, 0.61}
\definecolor{bole}{rgb}{0.47, 0.27, 0.23}
\definecolor{gris}{gray}{0.975}
%----------------------------------------------------------------------------------------------
% Si desea utilizar \@@line para definir su propio encabezado de examen o palabras del encabezado, 
% asegúrese de usar \makeatletter y \makeatother en los lugares apropiados, de lo contrario 
% podría obtener errores.
%-----------------------------------------------------------------------------------------------
\theoremstyle{plain}
\newtheorem{theorem}{Theorem}[section]
\newtheorem{thm}[theorem]{Teorema}
\newtheorem{cor}[theorem]{Corolario}
\newtheorem{lem}[theorem]{Lema}
\newtheorem{pro}[theorem]{Proposición}
\newtheorem{axs}[theorem]{Axiomas}
\newtheorem{axi}[theorem]{Axioma}
\theoremstyle{definition}
\newtheorem{exas}[theorem]{Ejemplos}
\newtheorem{exa}[theorem]{Ejemplo}
\newtheorem{defi}[theorem]{Definición}
\theoremstyle{remark}
\newtheorem{rmk}[theorem]{Observación}
\newtheorem{step}{Step}
\newtheorem{xca}[theorem]{Ejercicio}
\newtheorem{prob}[theorem]{Pregunta}
\newtheorem{rmks}[theorem]{Observaciones}
\newtheorem*{proofmt}{Prueba del Teorema Principal}
\usepackage[centerlast,small,sc]{caption}
\setlength{\captionmargin}{20pt}
\newcommand{\axref}[1]{Axioma~\ref{#1}}
\newcommand{\defref}[1]{\textbf{Definición}~\ref{#1}}
\newcommand{\coref}[1]{\textbf{Corolario}~\ref{#1}}
\newcommand{\thref}[1]{\textbf{Teorema}~\ref{#1}}
\newcommand{\lref}[1]{\textbf{Lema}~\ref{#1}}
\newcommand{\exaref}[1]{Ejemplo~\ref{#1}}
\newcommand{\xcaref}[1]{Ejercicio~\ref{#1}}
\newcommand{\rmkref}[1]{Observación~\ref{#1}}
\newcommand{\pref}[1]{\textbf{Proposición}~\ref{#1}}
\newcommand{\fref}[1]{Figura~\ref{#1}}
\newcommand{\tref}[1]{Tabla~\ref{#1}}
\newcommand{\cref}[1]{\textbf{Capítulo}~\ref{#1}}
\newcommand{\sref}[1]{\textbf{Sección}~\ref{#1}}
\newcommand{\aref}[1]{Apéndice~\ref{#1}}
\newcommand{\eref}[1]{Ecuación~\eqref{#1}}
\newcommand{\dref}[1]{Diagrama~\eqref{#1}}
\usepackage{makeidx}
\usepackage{tikz,tkz-tab}%
\usetikzlibrary{matrix,arrows,positioning,shadows,shadings,backgrounds,
	calc, shapes, tikzmark}
\usepackage{tcolorbox, empheq}%
\tcbuselibrary{skins,breakable,listings,theorems}

\tcbset{opteqC/.style={skin=beamer,colback=red!1!white}}
\newcommand{\celda}[2]{
	\begin{minipage}{#1mm}
		\centering
		\vspace{2mm}
		#2
		\vspace{2mm}
	\end{minipage}
}
\makeatletter
\begin{examtop}
	\@@line{\parbox{3in}{\classdata \\[0.5cm]
			\textcolor{upforestgreen}{\textbf{\underline{T.P.N$^\circ$}}~\fbox{\textsc{9}}} \examtype}
		%                  ^^^^^^
		\hfill
		\parbox{3in}{\normalsize \namedata}}
	\bigskip
\end{examtop}

\def\namedata{\textcolor{upforestgreen}{\textbf{Estudiante}}:\hrulefill \\[\namedata@vspace]
	\textcolor{upforestgreen}{\textbf{Curso y División}}: 1er año, I \\[\namedata@vspace]
	\textcolor{upforestgreen}{\textbf{Profesor}}: Ferreira, Juan David \\[\namedata@vspace]
	\textcolor{upforestgreen}{\textbf{Fecha de Entrega}}:\hrulefill 
}
% manual page 11        
\begin{keytop}%
	\@@line{\hfill \Huge\texttt{\textcolor{upforestgreen}{Respuestas 
				Trabajo Práctico N$^\circ$~\fbox{\textsc{9}}}} \hfill}
	\bigskip
\end{keytop}%
\makeatother
\examname{\textcolor{upforestgreen}{\underline{\textbf{}}}}

\SectionPrefix{\textcolor{upforestgreen}{\textbf{Sección \arabic{sectionindex}}.} \space}
\Fullpages
\ContinuousNumbering
\DefineAnswerWrapper{}{}
\NumberOfVersions{1}
\class{{\textcolor{upforestgreen}{\large\textbf{E.P.E.S. Nro 51 ``J. G. A.''}}\\[0.3cm]
		\textcolor{upforestgreen}{{\large \textbf{Matemática}}}\\[0.3cm]
	    \textcolor{upforestgreen}{\textbf{Producto y Cociente de Fracciones.}}}}

\begin{document}
	
	
	%-------------------------------             FILLIN       ------------------------%
	\begin{fillin}[title={Completamos los espacios vacíos:}, resetcounter=no, rearrange=no]
		
		Recordando que un cálculo mental es aquel que no se escribe en una forma de resolver, donde utilizamos gráficos,
		cálculos auxiliares u otras herramientas que servirán para dar una resolución al problema, resolvemos lo siguiente...
		\begin{question}
			Efectuar las siguientes calculo de suma, resta, multiplicaciones y divisiones de fracciones, simplificando lo más posible los resultados. Completamos los espacios vacíos con dichos resultados:
			\begin{enumerate}
				\item $\left( \frac{9}{8}+\frac{7}{3}\right)  \cdot \frac{8}{4}  =$\blank{ $\frac{2}{4}$ }.
				\item $\left( \frac{3}{5}+\frac{8}{4}\right)  \cdot \frac{5}{2}  =$\blank{ $\frac{5}{4}$ }.
				\item $ \frac{4}{6}-\frac{2}{12} \cdot \frac{9}{4} =$\blank{ $\frac{9}{4}$ }.
				\item $\frac{9}{5}\cdot \left( \frac{15}{3}-\frac{3}{5}\right) +\frac{9}{5} =$\blank{ $\frac{4}{7}$ }.
				\item $\left( \frac{13}{4}-\frac{5}{2}\right) : \frac{9}{5}=$\blank{ $\frac{3}{4}$ }.
				\item $\left( \frac{8}{3}-\frac{2}{4}\right):\frac{9}{15} =$\blank{ $\frac{5}{6}$ }.
			\end{enumerate}
		\end{question}
	\end{fillin}
	%--------------------------------------------------------------------------------------------%
	\begin{shortanswer}[title={Pensar cómo resolver las situaciones problemáticas aplicando fracciones...},
		rearrange=no,resetcounter=yes]
		
	    \begin{question}
	    	Don José dejó  $\displaystyle{\frac{3}{5}}$ de la pared para pintar con colores cálidos. Si pintará de color café  $\displaystyle{\frac{2}{3}}$ de lo destinado a los colores cálidos, ¿qué parte de la pared será de color café?.
	    	
	    	\hrulefill
	    	\begin{answer}
	    		contenidos de tu respuesta.
	    	\end{answer}
	    \end{question}
        
        \begin{question}
        	En una tienda hay $60$ botellas de agua de $\displaystyle{\frac{1}{4}}L$ cada una. ¿Cuántos litros de agua hay en total?.
        	
        	\hrulefill
        	\begin{answer}
        		contenidos de tu respuesta.
        	\end{answer}
        \end{question}
    
        \begin{question}
        	Martha tiene un negocio en el cual vende huevos empacados por docena. Por la mañana, uno de sus clientes por le pide solamente $\displaystyle{\frac{5}{6}}$ de docena y por la tarde, otro de sus clientes por le pide solamente $\displaystyle{\frac{1}{4}}$ de docena. ¿Cuántos huevos debe vendió Martha?
        	
        	\hrulefill
        	\begin{answer}
        		contenidos de tu respuesta.
        	\end{answer}
        \end{question}
	\end{shortanswer}
    %--------------------------------------------------------------------------------------------%
	\begin{endmatter}
		\centerline{\LARGE \textcolor{upforestgreen}{\textbf{Producto y Cociente de Fracciones:}}}
		\vspace{.2cm}
		\begin{tcolorbox}[colback=red!10!white, colframe=tealgreen, title=\textbf{Producto de Fracciones:}]
			Para multiplicar fracciones, se multiplica numerador con numerador y denominador con denominador. 
			\begin{equation}
			\frac{\textcolor{red}{a}}{\textcolor{blue}{b}}\cdot \frac{\textcolor{red}{c}}{\textcolor{blue}{d}}=
			\frac{\textcolor{red}{a}\cdot \textcolor{red}{c}}{\textcolor{blue}{b}\cdot \textcolor{blue}{d}}   =
			\frac{\textcolor{red}{numerador}\cdot \textcolor{red}{numerador}}{\textcolor{blue}{denominador}\cdot \textcolor{blue}{denominador}}
			\end{equation}
			\end{tcolorbox}
			%-------------------------------------------------------------------------------------------%
			\vspace{.1cm}
			\begin{exa}
				Veamos algunos ejemplos de multiplicación de fracciones:
				\begin{align*}
				\frac{3}{4}\cdot\frac{7}{5}   &= \frac{3\cdot 7}{4\cdot5}    = \frac{21}{20},                 &&&
				\frac{3}{5}\cdot\frac{6}{4}   &= \frac{3\cdot 6}{5\cdot 4}   = \frac{18}{20} = \frac{9}{10},
				\\[0.2cm]
				\frac{3}{7}\cdot\frac{1}{2}   &= \frac{3\cdot 1}{7\cdot 2}   = \frac{3}{14},                  &&&
				\frac{11}{3}\cdot\frac{4}{5}  &= \frac{11\cdot 4}{3\cdot 5}  = \frac{44}{15},
				\\[0.2cm]
				\frac{13}{4}\cdot\frac{5}{20} &= \frac{13\cdot 5}{4\cdot 20} = \frac{65}{80} = \frac{13}{16}, &&&
				\frac{5}{3}\cdot\frac{5}{4}   &= \frac{5\cdot 5}{3\cdot4}    = \frac{25}{12}.
				\end{align*}
			\end{exa}
			\vspace{.1cm}
			\begin{tcolorbox}[colback=red!10!white, colframe=tealgreen, title=\textbf{Cociente de Fracciones:}]
				Para dividir fracciones se invierte el divisor y luego se multiplican ambas fracciones
				\begin{align*}
				\frac{\textcolor{red}{a}}{\textcolor{blue}{b}}: \frac{\textcolor{blue}{c}}{\textcolor{red}{d}}=
				\frac{\textcolor{red}{a}}{\textcolor{blue}{b}}\cdot \frac{\textcolor{red}{d}}{\textcolor{blue}{c}}=
				\frac{\textcolor{red}{a}\cdot \textcolor{red}{d}}{\textcolor{blue}{b}\cdot \textcolor{blue}{c}}=\frac{\textcolor{red}{numerador}\cdot \textcolor{red}{numerador}}{\textcolor{blue}{denominador}\cdot \textcolor{blue}{denominador}}
				\end{align*}
			\end{tcolorbox}
			\vspace{.1cm}
			\begin{exa}
				Veamos algunos ejemplos de cociente de fracciones:
				\begin{align*}
				\frac{2}{4}:\frac{3}{5}  &= \frac{2}{4}\cdot\frac{5}{3} = \frac{2\cdot 5}{4\cdot 3} = \frac{10}{12} = \frac{5}{6}, &&&
				\frac{7}{5}:\frac{6}{4}  &= \frac{7}{5}\cdot\frac{4}{6} = \frac{7\cdot 4}{5\cdot 6} = \frac{28}{30} = \frac{14}{15},
				\\[0.2cm]
				\frac{5}{7}:\frac{5}{2}  &= \frac{5}{7}\cdot\frac{2}{5} = \frac{5\cdot 2}{7\cdot 5} = \frac{10}{35} = \frac{2}{7}, &&&
				\frac{12}{8}:\frac{8}{5} &= \frac{12}{8}\cdot\frac{5}{8}= \frac{12\cdot 5}{8\cdot 8}= \frac{60}{64} = \frac{15}{16},
				\\[0.2cm]
				\frac{9}{4}:\frac{3}{5}  &= \frac{9}{4}\cdot\frac{5}{3} = \frac{9\cdot 5}{4\cdot 3} = \frac{45}{12}=\frac{15}{4}, &&&
				\frac{10}{3}:\frac{5}{2} &= \frac{10}{3}\cdot\frac{2}{5}= \frac{10\cdot 2}{3\cdot 5}= \frac{20}{15}=\frac{4}{3}.
				\end{align*}
			\end{exa}
			
		\end{endmatter}
\end{document}